
\section{机器学习课程的学习体会与建议}
\subsection{项目总结}

本次项目的起源是算法课上的一道题目。在课后深入研究过后,我发现这其中好涉及到很多有趣的子问题值得研究。在查阅了相当多的文献之后,我逐渐形成了一个解决问题的思路,通过逐层将问题拆解和规约,我将这个一开始看上去有些棘手的问题转换成了可以逐步求解的问题。

在系统的实现上,我也遵循了我的思维上的分层,将不同问题封装到不同的类中,以此使整个系统更加便于理解,也便于修改和维护。在实现之中,我还增加了数据可视化的模块化功能,这样有助于使用者更进一步研究自己所采用的数据,从数据中得到更多有用的信息,同时也借助数据可视化更进一步理解其中算法的奥秘。

对于差分进化算法,我也是无意中了解到的。一开始我并没有意识到该算法可以用来求解分段线性回归的全局最优解。但是当我了解到分段线性回归可以归约为多维的线性回归,我意识到可以通过差分进化算法优化残差平方和,来使得我最终找到$\beta$的值,也就是最佳断点位置。

在学习到本项目设计的算法等知识外,我也了解到了非常多的有关Python和Numpy的奇淫巧计,可以用来快速的解决看似复杂的问题,这在实现层进一步释放了程序员的创造力,让程序员能够有更多精力关注较为上层的东西。我想这样优良的工具应该是计算机科学家或者数据科学家的福音吧。

总而言之,经历了这一个月的思考和摸索以及不断的查阅资料,这个我之前不太明白的问题现在对我来说已经相当简单,我对其中的原理也理解的非常透彻了。

\subsection{反思}

虽然本次项目的基本目的达到了,但是回首反思这给项目依旧存在诸多可以进一步改进和优化的地方。

首先,最大的一个可以优化的地方就是对于未知断点数目的多段线性回归可以给出一个更加令人满意的解决方案。虽然通过查阅文献,我了解到很多确认断点数目的方法,但是由于平时课业繁重,还有很多算法没有弄明白,因此也没有在系统中实现。

其次,系统中的一些小的细节还有提升的空间,比如数据集的奇异问题,可以通过预处理数据使得数据中不存在两个自变量一样的点来解决,但是由于时间紧急,我也没有来得及修改,只是将奇异异常加入程序,期待使用者能够自行对数据进行处理。

最后,本项目没有考虑自变量多维的情况。即,虽然前期的推导完全适用于多维数据,但是在系统具体实现的过程中,我发现多维数据还需要很多其他的考虑,而因为本项目时间紧急,所以没法把每个细节一一考虑到,因此系统也不支持多维数据的分段线性拟合。

我想,这些还可以优化的细节之处吸引着我进一步在本项目上精进。我想在未来的一段时间里,如果我能抽出时间,我就可以对我上述提的问题进行改进,使得系统整体更加完善。


\subsection{学习体会}

机器学习这门课是我上大学以来为数不多一些计算机专业的理论实践课。其中涉及到非常多的数学原理和统计学知识,具有理论性,同时机器学习的诸多算法都是非常巧妙高效的,在现实生活中有着非常多应用。

这门课带我领略到潜藏在我们生活中的那些神奇的地方,很多我们习以为常的神奇应用的背后都是有强大的机器学习算法或其它人工智能算法支撑的。也就是说,机器学习这门课如同向我讲述每一个魔法背后的奥秘,带领我掌握其中的奥义,并创造属于自己的神奇应用。

虽然在最后的项目中,我实现的都是一些非常基础的应用,但我想这是我学习机器学习的入门重要一步,在未来我还将继续在这个领域不断钻研学习,争取创造自己神奇应用。

在这门课中我也进一步了解了算法之禅。我一直认为,机器学习和算法是不分家的,机器学习同样包含于算法之中。通过学习机器学习,我了解到算法作为人类智慧结晶的美妙之处,这深深的吸引着我不断向更深处迈进。


\subsection{建议}

在学完这门机器学习课程之后,我从学生角度对这门课有了一种体验和认知。如果说,这门课还有哪些可以补充的话,我认为,还可以增加最后大项目的选题数。虽然本项目有第三题自选题,但是在缺乏老师的引导下,我自主探索这个课题其实比较艰难,很多时候我还需要找到老师对我加以引导和指正。所以,如果能增加选题数,老师能正面的给予同学们更多的探索方向,我认为是非常好的,这有利于激发同学嗯的探索热情促进同学们更好掌握机器学习模型。
